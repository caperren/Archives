\subsection{Corwin Perren}
\subsubsection{Fall}
\begin{itemize}
\item Week 1
	\begin{itemize}
	\item Week 1.3
      \begin{itemize}
      \item Looked at and selected projects for my preference submission. 
      \end{itemize}
    
    \item Summary
      \begin{itemize}
      \item Looked through potential projects and submitted project preferences. I also contacted and got a positive confirmation from Nick McComb, the stakeholder for the OSURC Mars Rover capstone project, that myself and Christopher Pham were requested to work on the project via an email to McGrath.  
      \end{itemize}
	\end{itemize}

\item Week 2
	\begin{itemize}
	\item Week 2.2
      \begin{itemize}
      \item Took class notes
      \end{itemize}
	\item Week 2.3
      \begin{itemize}
      \item Group wrote an email to send to our client, Nick
      \item Sent said email to Nick 
      \end{itemize}
	\item Week 2.7
      \begin{itemize}
      \item Had a meeting with Nick McComb, Kenneth, and Chris to covered the design guidelines document 
      \end{itemize}
    
    \item Summary
      \begin{itemize}
      \item Met with group members Chris and Kenneth, where we exchanged contact information and coordinated contacting our client to set up a meeting. We made contact and set up a video conference call for Sunday to cover project details. We also were added to the rover team's slack, which we'll use for primary communication in the future. For future use, we were also added to the club's GitHub account. 
      \end{itemize}
	\end{itemize}

\item Week 3
	\begin{itemize}
	\item Week 3.1
      \begin{itemize}
      \item Worked on and completed the rough draft problem statement 
      \item Did some initial research on the tools and frameworks our team will be using for this project 
      \end{itemize}
	\item Week 3.2
      \begin{itemize}
      \item Started communication as a group with our TA to set up a weekly meeting time. 
      \end{itemize}
    
    \item Summary
      \begin{itemize}
      \item After communicating with our client on Sunday, I took Sunday night and Monday to work on and complete the problem statement document. I also began to look more into the tools and frameworks we will be using on this project. Our group also started emailing our TA, Andrew, to set up a weekly meeting time, which is still not set in stone as Andrew is having a hard time scheduling so many groups at once. So far, it looks like they'll probably be on Tuesday afternoons as a remote session. 
      \end{itemize}
	\end{itemize}

\item Week 4
	\begin{itemize}
	\item Week 4.4
      \begin{itemize}
      \item Made revisions to the problem statement document 
      \end{itemize}
	\item Week 4.5
      \begin{itemize}
      \item Uploaded the final version of the document to the Github rep
      \item Emailed client asking for revisions, and an approval email once ready 
      \item Embedded PDF in the group OneNote 
      \end{itemize}
    
    \item Summary
      \begin{itemize}
      \item Our group used overleaf to group edit a join problem statement document, made revisions, and then emailed our client the final version. We're currently waiting for the approval email. I also embedded our PDF into the group OneNote per the follow-up email from Kirsten 
      \end{itemize}
	\end{itemize}

\item Week 5
	\begin{itemize}
	\item Week 5.2
      \begin{itemize}
      \item Had TA Meeting
      \end{itemize}
	\item Week 5.4
      \begin{itemize}
      \item Worked on design requirements document with Chris and Ken 
      \end{itemize}
    
    \item Summary
      \begin{itemize}
      \item We had a general meeting with our TA, the first of the term. Covered what the TA's role will be in our group. Covered that we seem to have a good plan on what our group will be doing. On Thursday, worked on the design requirements rough draft.  
      \end{itemize}
	\end{itemize}

\item Week 6
	\begin{itemize}
	\item Week 6.2
      \begin{itemize}
      \item Had an update meeting with the TA 
      \end{itemize}
	\item Week 6.3
      \begin{itemize}
      \item Worked on design requirements document with Chris and Ken 
      \end{itemize}
	\item Week 6.4
      \begin{itemize}
      \item Continued working on design requirements document with Chris and Ken 
      \item Sent the semi-final draft to the client for potential edits 
      \end{itemize}
	\item Week 6.5
      \begin{itemize}
      \item Made client edits to requirements document 
      \item Sent approval request email to client 
      \end{itemize}
    
    \item Summary
      \begin{itemize}
      \item We had a general meeting with our TA and otherwise spent the week working on our requirements document. We sent a semi-final draft to our client Nick, who requested some minor edits, which we made and then sent him a final draft for approval.  
      \end{itemize}
	\end{itemize}

\item Week 7
	\begin{itemize}
	\item Week 7.2
      \begin{itemize}
      \item Decided on projects to work on 
      \item Had TA meeting
      \end{itemize}
	\item Week 7.4
      \begin{itemize}
      \item Set up the ground station hardware in Graf 306 
      \end{itemize}
    
    \item Summary
      \begin{itemize}
      \item Last week, we had a short quick meeting with our TA and then worked with each other to determine who would take what technologies for the tech review document. On Thursday, I set up the hardware our team will be running the ground station on set up in Graf 306. This included an Intel NUC, two joysticks, and two 1080p monitors. For now, it's connected via and ethernet switch to a desktop emulating the Rover. 
      \end{itemize}
	\end{itemize}

\item Week 8
	\begin{itemize}
	\item Week 8.1
      \begin{itemize}
      \item Worked on rough draft for the tech review document 
      \end{itemize}
	\item Week 8.2
      \begin{itemize}
      \item Did peer revisions on tech review documents in class 
      \item Turned in tech review 
      \item Met with TA
      \end{itemize}
	\item Week 8.5
      \begin{itemize}
      \item Worked on setting up the ROS framework on ground station software 
      \item Got simple rover drive control data sending to the rover 
      \item Got three video streams showing on the gui from the rover 
      \end{itemize}
    
    \item Summary
      \begin{itemize}
      \item Last week, I finished my peer review, did an in class peer review, and turned it in. Our group also had another short meeting with our TA to make sure we were doing fine. On Friday, I began writing some initial ground station code. I got simple rover drive data sending over the network to the Rover. I also got three camera streams displaying in the GUI. 
      \end{itemize}
	\end{itemize}

\item Week 9
	\begin{itemize}
	\item Week 9.1
      \begin{itemize}
      \item Worked on the final draft for the tech review 
      \end{itemize}
    
    \item Summary
      \begin{itemize}
      \item I finished the final draft of my tech review on Monday night and submitted it mid-day Tuesday.  
      \end{itemize}
	\end{itemize}

\item Week 10
	\begin{itemize}
	\item Week 10.2
      \begin{itemize}
      \item Meeting with TA
      \end{itemize}
	\item Week 10.3
      \begin{itemize}
      \item Created latex template doc on overleaf 
      \item Added initial content
      \end{itemize}
	\item Week 10.4
      \begin{itemize}
      \item Worked on final draft of the design document 
      \end{itemize}
	\item Week 10.5
      \begin{itemize}
      \item Worked on final draft of the design document 
      \item Turned in the final draft of the design document 
      \end{itemize}
    
    \item Summary
      \begin{itemize}
      \item I worked on and finished the final draft of our group design document and turned it int. We also had a 60 second meeting with our TA this week.  
      \end{itemize}
	\end{itemize}

\end{itemize}




\subsubsection{Winter}
\begin{itemize}
\item Week 1
	\begin{itemize}
	\item Week 1.2
      \begin{itemize}
      \item We went and checked out our lab space in Graf 306 
      \item A work schedule was made to work at least five hours on Tuesday, Thursdays, and Saturdays, with the Saturday work overlapping with the Mars Rover weekly meetings 
      \item Planned to start actual development work Thursday 
      \end{itemize}
	\item Week 1.4
      \begin{itemize}
      \item I filled out Asana with the core tasks that our software need to produce 
      \item Initial tasks were divvied up 
      \item I began continuing my work on processing and displaying video data 
      \end{itemize}
    
    \item Summary
      \begin{itemize}
      \item Chris, Ken, and I met up after class on Tuesday so I could show them our lab space in Graf 306. We then came up with a work schedule for the term where we work on Tuesdays, Thursdays, and Saturdays for five hours a day at minimum. Then on Thursday, we all began actual work starting with creating tasks to complete and assigning initial tasks to everyone. I then continued working on the video display systems. I made the processing and display code slightly more efficient, and also added stream labels and fps counters directly on the video stream. On Saturday, I plan to add in auto adjustment of the stream quality based on whether the fps values are too low. Conversely, if the fps is pegged at the maximum, the quality will be increased until it starts to drop. 
      \end{itemize}
	\end{itemize}

\item Week 2
	\begin{itemize}
	\item Week 2.2
      \begin{itemize}
      \item Continued working on video systems 
      \end{itemize}
	\item Week 2.4
      \begin{itemize}
      \item Began refactoring video receiving class after verifying that the core class worked well. 
      \end{itemize}
	\item Week 2.6
      \begin{itemize}
      \item Began refactoring the core of the ground station launcher file and restructured the program folder to have a cleaner layout. 
      \end{itemize}
    
    \item Summary
      \begin{itemize}
      \item Over that week, I mainly worked on the video receiving class, testing it to make sure it was reliable and could show enough video streams at a smooth frame rate. Once that was verified, I began refactoring the class into a new version that would be able to include turning on and off video streams and adjusting what streams were shown in what display box. Later in the week, I began refactoring the ground station launcher file and folder structure so we'll have a cleaner base to work with as the project progresses. 
      \end{itemize}
	\end{itemize}

\item Week 3
	\begin{itemize}
	\item Week 3.2
      \begin{itemize}
      \item Finished refactor of the launcher for the ground station 
      \item Added in check for if the rover is on the network when the software starts 
      \item Continued refactor of the video classes 
      \end{itemize}
	\item Week 3.4
      \begin{itemize}
      \item Worked on adding advanced features of the video class refactor. 
      \item Added sleeping of video streams 
      \item Made number of cameras and resolutions dynamic 
      \item Can handle smart changing of what video displays are in each GUI element 
      \end{itemize}
    
    \item Summary
      \begin{itemize}
      \item Over this week, I finished the refactor of the launcher for the ground station software. This included an addition to keep the program from launching if the Rover is not present on the network. Also, there is now a shared object that can be passed to all the important classes in the program to allow for easy interfacing with signals from other classes and the GUI. I also made significant progress on the Video class refactoring. At the moment, the GUI can now display a dynamic number of video streams, and also allows for dynamically changing which stream is displayed in each video display element. I've also added the ability to disable video streams where it now actually stops pulling in the video data to lower bandwidth usage. This includes "sleeping" video streams that aren't explicitly disabled, but are not actively being shown. Lastly, I finished the core of the GUI layout using QT Designer so the core widget modules now match the layouts defined in our documents from last term. 
      \end{itemize}
	\end{itemize}

\item Week 4
	\begin{itemize}
	\item Week 4.2
      \begin{itemize}
      \item Continued work on video systems 
      \item Assisted main software team with their GPS and IMU firmware, as well as creating a ROS node to broadcast said data. Goal is to get GPS data sooner so Chris has that data to work with. 
      \end{itemize}
	\item Week 4.4
      \begin{itemize}
      \item Assisted with firmware programming for drive control micro-controllers 
      \item Assisted on work on the motor controllers drive nodes for ROS so our team can have a node to send drive commands to. 
      \end{itemize}
    
    \item Summary
      \begin{itemize}
      \item This week was mostly spent helping with development on software packages that will help support the ground station software on the Rover. Our group is getting to the point where having working ROS nodes on the Rover and hardware to control is getting more important, so I decided to help speed up the process. I worked on firmware for the micro-controllers that processed raw GPS and IMU data as well as the motor controller. I then also helped write the ROS nodes to interact with these micro-controllers so they could be controller using normal ROS topics. 
      \end{itemize}
	\end{itemize}

\item Week 5
	\begin{itemize}
	\item Week 5.2
      \begin{itemize}
      \item Took a break from capstone software to assemble revision 1 of the Mars Rover Iris control board 
      \item Spent 10 hours assembling Iris 
      \end{itemize}
	\item Week 5.4
      \begin{itemize}
      \item Spend 7 hours testing and fixing electrical problems with the Iris board 
      \item Ending board had full functionality on the 12 communication ports 
      \end{itemize}
    
    \item Summary
      \begin{itemize}
      \item This week was pretty light for capstone as I had midterms. Additionally, I decided to take some time to help the Mars Rover electrical team, and assembled their Iris central control board. Further development of the ground station will be greatly helped by having missing hardware finally connected to the Rover to talk to, so I figured spending some extra time helping this to happen would be useful for the team overall, and then by proxy, our capstone team. Assembly, testing, and modifications took 17 hours. I also wrote a simple motor driver node for ROS so the motor driver board could be tested using control from ROS topics. 
      \end{itemize}
	\end{itemize}

\item Week 6
	\begin{itemize}
	\item Week 6.2
      \begin{itemize}
      \item Worked on and completed rough draft of expo poster 
      \item Helped Chris with mapping integration 
      \item Wrote simple joystick control file to be run on ground station 
      \item Added UI elements 
      \end{itemize}
	\item Week 6.4
      \begin{itemize}
      \item Worked on midterm report 
      \end{itemize}
    
    \item Summary
      \begin{itemize}
      \item This week was spent mostly working on administrative stuff for the capstone project. A rough draft of the expo poster was made, and the later part of the week was spent working on the midterm report and video. However, I also did create a simple test file that takes in joystick data on the ground station, and broadcasts it to a ROS topic, allowing for remote driving of the motor connected to the Rover through Iris. 
      \end{itemize}
	\end{itemize}

\item Week 7
	\begin{itemize}
	\item Week 7.4
      \begin{itemize}
      \item Helped with SAR work 
      \item Made multi-threaded controller code. 
      \item Tested nimbo\_sender / receiver 
      \end{itemize}
    
    \item Summary
      \begin{itemize}
      \item This week I helped out getting things ready for the Rover systems acceptance review, one of the pivotal turning points for the rover team. This involved cleaning up code and getting remote drive systems working. As part of this, I re-wrote the joystick control test file to be multi-threaded and also got drive working with nimbro\_transport, which allows for UDP sending / receiving of ROS topics for packet loss allowable topics such as drive and video. 
      \end{itemize}
	\end{itemize}

\item Week 8
	\begin{itemize}
	\item Week 8.2
      \begin{itemize}
      \item Continued work for SAR deadline 
      \item Wrote drive sender so that sent control data matched the refactored drive\_receiver on the Rover 
      \item Joystick control speed limiting using throttle axis, shows speed limit and tank output on GUI. 
      \item Tuned drive sender until remote drive over radio was smooth 
      \item Helped with refactor of whole GitHub structure 
      \item Needed for easy access to shared ROS packages due to nimbro addition 
      \item Made major updates to GUI design, moved closer to matching design document.  
      \item Arm joint position placeholders 
      \item Rover status placeholders 
      \item Added pause feature to drive controller, starts in pause by default 
      \item Added mock compass image to the GUI, for show during SAR video 
      \item Tested remote drive and video with no tether! Drove around parking lot from ground station. 
      \end{itemize}
	\item Week 8.3
      \begin{itemize}
      \item Helped create videos for SAR demo 
      \end{itemize}
    
    \item Summary
      \begin{itemize}
      \item Due to the SAR deadline being March 2, I put in extra time this week to make systems work as well as they could for the demo video made on Wednesday. For the demo, we wanted at minimum remote Rover drive working, video systems buttoned up and working well. On this note, wrote a joystickcontrolsender class to match the new rover\_drive system implemented on the rover. This code was the finalized version of the testing joystick code I'd written. I also added the ability to use the throttle stick on the joystick to act as a speed limit for the drive commands sent. So a setting of 50\% would scale the full joystick range to only 50\% speed. Additionally, a pause feature was added so that the rover cannot be accidentally controlled when the GUI first starts up. A button on the joystick disables pause state, and can be used to toggle the state while the gui is running. Also, gui elements on the right monitor were updated to show the speed limit value and the current power output to each of the sides of the rover in tank drive form. After making these changes, I tuned update rates and nimbro transport rates to get driving smooth again.  

Due to the addition of nimbro\_network, I refactored the GitHub software repository to make it easier to share packages between the rover and ground station software. This was needed so that both had access to custom designed message formats in ROS. I also worked a lot this week to update the look of the GUI to match our design document. Placeholders images or Qt GUI elements for arm joint positions, rover statuses, and the compass were all added.  

On Tuesday night, I and the team's mechanical lead, Ben, tested actual real remote drive of the Rover. We placed the rover in the Graf parking lot where he followed it with a kill switch tether, and I drove the rover over the remote radio link from the 3rd floor of Graf with the antenna sticking out the window. Tests showed that overall everything worked well! Driving was mostly smooth and video responsive enough to feel comfortable driving. Then on Wednesday, we remotely drove the rover on the catwalk of 3rd floor Graf and took videos of both the Rover driving, and of me controlling it at a now largely flushed out looking ground station. 
      \end{itemize}
	\end{itemize}

\item Week 9
	\begin{itemize}
	\item Week 9.2
      \begin{itemize}
      \item Tuned some parameters of the drive controller so that driving was a bit smoother after remote drive testing the previous week. 
      \item Rewrote video\_receiver and coordinator to include new topic to control which video resolution is enabled, side effect of nimbro 
      \item Added auto resolution adjustment of video\_receivers 
      \item Tested new end effector video stream 
      \end{itemize}
	\item Week 9.4
      \begin{itemize}
      \item Added joystick control sending and receiving for selection of cameras in GUI 
      \item Helped Chris integrate full map system into master branch 
      \item Added the ability to move compass, just spinning never-ending in one direction. 
      \end{itemize}
    
    \item Summary
      \begin{itemize}
      \item After testing real remote driving the previous week, I tuned parameters on the rover and the ground station to get remote drive working just a little bit smoother and with no dropouts. After that, I worked on rewriting my video\_receiver and video\_coordinator threads so that instead of simply opening the designed video stream, the ground station would tell the rover which stream to send. This was a side effect of using the UDP nimbro\_network transports. Because UDP continually spams data at the ground station, it was in effect sending all the different resolution steams to the ground station all the time. This extra bandwidth was definitely something the team didn't want. During the rewrite, I also added automatic resolution adjustment of the video\_receivers based on averages of the received frame rates.  

I also tested and integrated receiving and showing the fourth and final video stream from the end\_effector, which worked as intended. Receiving of rover\_statuses was also integrated into the roslaunch file and I tested receiving those statuses from command line. I ended up adding and additional manual update message for that package on the rover so the ground station could request and initial state when the software first starts up, and tested this via rostopic publish from the command line.  

As I'd gotten the joystick controlclass cleaned up the previous week, I also wrote and tested joystick control of which video streams were shown on the three gui elements while I was rewriting the receivers and video coordinators. Two sets of two buttons each on the joystick now cyclically change which element is selected via an orange ring around the outside as well as what camera stream is current shown. The trigger on the joystick disables the currently selected stream. 

I also helped integrate Chris' mapping system into the master branch of the rover software before ending the week by writing the threaded class needed to make the compass spin. I ended by making the compass spin in one direction indefinitely. 

I sat down in the middle of the week and helped the mechanical team lead determine the final design for the quick deployment ground station box, which he then began fabricating. 
      \end{itemize}
	\end{itemize}

\item Week 10
	\begin{itemize}
	\item Week 10.2
      \begin{itemize}
      \item Made nice new compass image asset as vector image in Inkscape, exported as PNG 
      \item Compass now can respond to heading updates (once we get them). For now, left/right clicks change heading 
      \item Heading and next goal heading update 
      \item Arm joint positions spin/stop when clicked. Placeholders. 
      \item Help Ken integrate his system statuses with the GUI 
      \item Placeholders for waypoint entry/editing 
      \item Wrote test code to get info from the M2 radios, set channel 
      \item Refactored M2 radio test code into one module to set radio channel from GUI, added GUI button and setting. Tested and works. 
      \item Helped Ken debug thread context issues updating GUI 
      \item Added pitch and roll indicators for Rover for when we get that data 
      \end{itemize}
	\item Week 10.4
      \begin{itemize}
      \item Made minor changes to statuses so that it looked nicer 
      \item Added OSURC logo to top left corner of GUI per Nick's request 
      \item Made numerous UI changes to help clean up look. 
      \item Refactored joysticksystems to inputsystems 
      \item Added spacenavsystems to read in spacenav mouse data 
      \end{itemize}
    
    \item Summary
      \begin{itemize}
      \item This final week was spent mostly dealing with non-critical updates to the look and feel and glue logic for the GUI. I made a brand new compass asset in inkscape that looks like a real compass and exported it as a high resolution PNG to be imported in the GUI. I then finished off the compass class allowing for the compass to rotate to a heading if one is received from the Rover. As the rover is not currently broadcasting this data, left and right clicking on the GUI element causes it to add or subtract five degrees from a mock heading, whose update is then shown. Additionally, I also made the textual readouts show the current heading to prove we have control of them. I did something similar with the arm joint positions by making them all spin when left clicked and stop spinning when right clicked. These are placeholders until we get arm data later next term.  

Ken got his system statuses class ready to integrate into the main GUI and launcher, so I helped him do that and make the minor changes needed to make his thread start and stop properly. I also helped him debug a thread context issue that was causing weird flickering on the status display elements. Afterwards, I added the gui elements for manually entering GPS coordinates for adding to the waypoints listings.  

I wrote test code to interact with the Rocket M2 radios we're using to get the status data we'll end up needing, such as signal strength, channel, and throughput. I also got test code working to set the 2.4GHz channel that the radios are set to. Afterwards, I wrote a class to handle the channel updates, made gui elements to control it, added this class to the GUI, and tested it. Now, a channel from 1-11 can be set from the GUI. After pressing the button, radios are reconnected and data returns after 4-10 seconds.   

Near the end of the week, I made mostly graphical changes to the GUI such as adding and OSURC logo to the top left corner, and adjusting spacing and layouts to be more pleasing. I ended this week by refactoring the joysticsystems package into inputsystems before adding the spacenavsysystem class. I currently have this class reading in and storing SpaceNav mouse data in a class variable, waiting to be broadcast to appropriate systems once they're ready to accept the control data. 

The mechanical team also showed us the assembled quick-deployment box, which should be ready for painting and then hardware install in the next few days. 
      \end{itemize}
	\end{itemize}

\end{itemize}




\subsubsection{Spring}
\begin{itemize}
\item Week 1
	\begin{itemize}
	\item Week 1.2
      \begin{itemize}
      \item Talked with group mates over slack about what needs to be completed this term 
      \item Generally talked about the fact that the team was changing competitions 
      \item Didn't work on the project as everyone was busy 
      \end{itemize}

    \item Summary
      \begin{itemize}
      \item Chris, Ken, and I chatted over slack about what needs to be done over the term. Not much is left, and a lot of the work is waiting on Rover hardware. Also, we talked about the fact that the Rover team has switched to the Canadian International Rover Challenge. 
      \end{itemize}
	\end{itemize}

\item Week 2
	\begin{itemize}
	\item Week 2.4
      \begin{itemize}
      \item Did a work day on Thursday, I worked on SpaceNav mouse integration. 
      \item I registered the team for EXPO 
      \item I also dealt with sending in expo release forms 
      \item We made a list of all tasks left to be completed 
      \item Went over design requirements and found a few that need to be modified due to Rover design changes over the year. 
      \end{itemize}
    
    \item Summary
      \begin{itemize}
      \item We all met on Thursday for a work day. I worked on continued integration of the SpaceNav mouse, getting values normalized and broadcasting to Chris could use it to pan the map. I also registered the team for EXPO and dealt with sending in the media release forms for our team. I also went over the design requirements document with Chris and Ken and found some elements that need to be changed as they're either no longer pertinent, or have changed enough that it would be better to update their descriptions. We'll be making these changes in the following week. Part of this was looking at the differences in the rules for University Rover Challenge vs the Canadian competition the team is now going to.  
      \end{itemize}
	\end{itemize}

\item Week 3
	\begin{itemize}
	\item Week 3.4
      \begin{itemize}
      \item Worked on Rover side software to help get more statuses for Ken 
      \item Made new and worked on new rover package rover\_arm to get rover arm integrated with ROS 
      \end{itemize}
    
    \item Summary
      \begin{itemize}
      \item This week I worked on Rover code rather than ground station code as we're now mostly waiting on features of the Rover to be finished before we can implement the appropriate ground station side code. This involved me getting rover battery status monitoring working and sending to the ground station as well as helping another member get GPS statuses doing the same. I also made and began work on the Rover's rover\_arm package to interface with the IONI motor controllers used to move the arm. I got a simple package compiling with both the ROS packages and simplemotion library and got an arm motor moving with it. 
      \end{itemize}
	\end{itemize}

\item Week 4
	\begin{itemize}
	\item Week 4.2
      \begin{itemize}
      \item Finished final draft of poster 
      \end{itemize}
	\item Week 4.6
      \begin{itemize}
      \item Worked on and submitted a revised version of our design requirements document. 
      \end{itemize}
    
    \item Summary
      \begin{itemize}
      \item This week, I worked with my team to make the final changes to our poster for expo. I then submitted the poster for printing. I also worked with my team to make modifications to our design requirements document to reflect the team's change to the Canadian International Rover Challenge. We got the revised document signed off by all group members and the stakeholder and emailed it to the professor, TA, and stakeholder. 
      \end{itemize}
	\end{itemize}

\item Week 5
	\begin{itemize}
	\item Week 5.6
      \begin{itemize}
      \item Worked on midterm progress report 
      \end{itemize}
    
    \item Summary
      \begin{itemize}
      \item This was a busy midterm week, so I just worked on the midterm report and presentation. 
      \end{itemize}
	\end{itemize}

\item Week 6
	\begin{itemize}    
    \item Summary
      \begin{itemize}
      \item No work was done this week 
      \end{itemize}
	\end{itemize}

\item Week 7
	\begin{itemize}
	\item Week 7.5
      \begin{itemize}
      \item Went to expo! 
      \end{itemize}
    
    \item Summary
      \begin{itemize}
      \item Our team was at expo! Things went well! The demo was smooth. 
      \end{itemize}
	\end{itemize}

\item Week 8
	\begin{itemize}
	\item Week 8.6
      \begin{itemize}
      \item Got the ground station controlling the pan/tilt node
      \end{itemize}
    
    \item Summary
      \begin{itemize}
      \item I finished the firmware for a prototype pan/tilt node and wrote the code for the GUI to control it using the joystick. Tested and worked! 
      \end{itemize}
	\end{itemize}

\item Week 9
	\begin{itemize}
    \item Summary
      \begin{itemize}
      \item No work was done this week 
      \end{itemize}
	\end{itemize}

\item Week 10
	\begin{itemize}
	\item Week 10.2
      \begin{itemize}
      \item Started final report document 
      \end{itemize}
	\item Week 10.4
      \begin{itemize}
      \item Worked on final video presentation. 
      \item Recorded my portion of presentation. 
      \end{itemize}
    
    \item Summary
      \begin{itemize}
      \item This week, I've worked on the final video and started the final report document to finish off capstone. 
      \end{itemize}
	\end{itemize}

\end{itemize}
\section{QEMU Flags}

\begin{itemize}
  \item \texttt{-gdb tcp::????}

    Waits for a GDB connection on the specified device, namely a TCP
    connection on port ????.

  \item \texttt{-S}

    Prevents the CPU from starting up when first launching the command.

  \item \texttt{-nographic}

    Disables graphical capabilities and turns qemu into a command line
    application.

  \item \texttt{-kernel bzImage-qemux86.bin}

    Uses the ``\texttt{bzImage-qemux86.bin}'' binary as the kernel image to boot.

  \item \texttt{-drive file=core-image-lsb-sdk-qemux86.ext4,if=virtio}

    Uses ``\texttt{core-image-lsb-sdk-qemux86.ext4}'' as the file for the virtual
    hard drive, and virtio as the I/O interface (virtio virtualizes I/O
    operations like disk reads/writes).

  \item \texttt{-enable-kvm}

    Enables use of kernel-based virtual machine technology.

  \item \texttt{-net none}

    Indicates that no network devices should be configured.

  \item \texttt{-usb}

    Enables USB drivers.

  \item \texttt{-localtime}

    Uses the local time from the host machine to set the time in the guest,
    instead of the default of UTC time.

  \item \texttt{--no-reboot}

    Shuts down the guest entirely on exit instead of rebooting.

  \item \texttt{--append "root=/dev/vda rw console=ttyS0 debug"}

    Uses \texttt{"root=/dev/vda rw console=ttyS0 debug"} as the initial command line,
    running it on initial boot.
\end{itemize}